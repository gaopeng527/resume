% !TEX TS-program = xelatex
% !TEX encoding = UTF-8 Unicode
% !Mode:: "TeX:UTF-8"

\documentclass{resume}
\usepackage{zh_CN-Adobefonts_external} % Simplified Chinese Support using external fonts (./fonts/zh_CN-Adobe/)
%\usepackage{zh_CN-Adobefonts_internal} % Simplified Chinese Support using system fonts
\usepackage{linespacing_fix} % disable extra space before next section
\usepackage{cite}

\begin{document}
\pagenumbering{gobble} % suppress displaying page number

\name{高健人}

\basicInfo{
  \email{gaopeng527@163.com} \textperiodcentered\
  \phone{(+86)17338126815} \textperiodcentered\
  \github{https://github.com/gaopeng527}}
 
\section{\faGraduationCap\  教育背景}
\datedsubsection{\textbf{大连理工大学}, 辽宁}{2014.09 -- 2017.06}
\textit{硕士研究生}\ 计算机应用技术(排名:前30\%)
\datedsubsection{\textbf{河南科技大学}, 河南}{2010.09 -- 2014.06}
\textit{学士}\ 数学与应用数学(排名:前5\%)

\section{\faCogs\ 专业技能}
% increase linespacing [parsep=0.5ex]
\begin{itemize}[parsep=0.5ex]
  \item 熟练掌握Flink大数据流式计算框架的使用及参数调优,有相关项目开发经验
  \item 熟练掌握HBase海量数据存储行键设计及查询优化策略
  \item 熟练掌握kafka消息队列,熟练运用Zookeeper
  \item 熟练掌握应用服务器软件Tomcat、Nginx等容器配置和部署,熟悉Linux开发环境
  \item 精通Java面向对象编程,熟练掌握数据结构与算法
  \item 熟练运用SpringMVC,SpringBoot,Spring,MyBatis等Java常用Web开源框架
  \item 掌握Spark和Hadoop大数据开源框架的使用
  \item 掌握dubbo分布式服务框架的搭建及使用
\end{itemize}

\section{\faUsers\ 工作履历}
\datedsubsection{北京嘀嘀无限科技发展有限公司}{2017.07 -- 2019.08}
\datedsubsection{北京联行网络科技有限公司}{2019.08 -- 至今}

\section{\faUsers\ 项目经历}
\datedsubsection{\textbf{数据中台构建}}{2019.08 -- 2020.02}
\role{\textbf{公\hspace{2em}司:}}{北京联行网络科技有限公司}
\role{\textbf{开发环境:}}{IDEA + Linux + Flink + Kafka + Mysql + SpringBoot + Git}
\role{\textbf{责任描述:}}{实时计算负责人}
\textbf{项目简介:}
\begin{itemize}
  \item 将公司充电业务产生的各类数据进行分层聚合处理,构建数仓以统一的方式提供给公司其他部门使用,比如风控拦截、运营分析等
\end{itemize}
\textbf{个人成果:}
\begin{itemize}
  \item 完成基本的Flink实时计算平台搭建,并实现了一套以SQL引擎方式处理binlog日志、json数据的通用框架,极大提高了数据处理效率
  \item 规范化整个端上埋点数据接入及处理流程,并提供实时查询能力
  \item 建立通用的报表系统,支持以sql方式配置表格、折线图、柱状图、饼图、单指标等各种组件的创建及组合报表展示,并提供告警功能
\end{itemize}

\rule{\textwidth}{0.1mm}
\datedsubsection{\textbf{新出租政企SaaS平台}}{2019.04 -- 2019.07}
\role{\textbf{公\hspace{2em}司:}}{北京嘀嘀无限科技发展有限公司}
\role{\textbf{开发环境:}}{IDEA + Linux + Flink + HBase + SpringBoot + Git}
\role{\textbf{责任描述:}}{后端主要研发人员}
\textbf{项目简介:}
\begin{itemize}
  \item 通过整合线上和线下出租车数据助力政府和企业进行车辆监控、服务管控、安全监管、公司管理和运营分析
\end{itemize}
\textbf{个人成果:}
\begin{itemize}
  \item 目前已完成线上出租车车辆监控和服务管控两个模块的主要功能开发
\end{itemize}

\rule{\textwidth}{0.1mm}
\datedsubsection{\textbf{TPCC全流程监控中心}}{2018.04 -- 2019.03}
\role{\textbf{公\hspace{2em}司:}}{北京嘀嘀无限科技发展有限公司}
\role{\textbf{开发环境:}}{IDEA + Linux + Flink + HBase + SpringBoot + Git}
\role{\textbf{责任描述:}}{业务大盘、事件引擎负责人}
\textbf{项目简介:}
\begin{itemize}
  \item 针对用户差体验的场景也越来越多、问题发现滞后、问题分析低效、问题改进效果回归缺乏真实性这些问题,建立全流程监控中心旨在提供问题发现、问题分析这样一种能力,帮助业务方快速发现分析相关问题,同时提供真实效果的验证机制,形成体验优化闭环,不断提升用户体验,主要分为CPO大盘、极端异常看板、业务大盘和交易流监控四个主要部分
\end{itemize}
\textbf{个人成果:}
\begin{itemize}
  \item 业务大盘目前已经构建了支付、奖励、券BadCase、听单少四个大盘,帮助业务方有效降低了相关进线情况,为公司每周节约成本约13万元,帮助团队提前发现问题、快速拦截进线,达3次以上
  \item 事件引擎可通过配置化的形式为交易流监控提供稳定的事件下发服务,以通用化的形式适配各种场景
\end{itemize}

\rule{\textwidth}{0.1mm}
\datedsubsection{\textbf{ODP流式计算}}{2017.07 -- 2019.03}
\role{\textbf{公\hspace{2em}司:}}{北京嘀嘀无限科技发展有限公司}
\role{\textbf{开发环境:}}{IDEA + Linux + HBase + Flink + SpringBoot + Git}
\role{\textbf{责任描述:}}{项目主要负责人}
\textbf{项目简介:}
\begin{itemize}
  \item 对公司规范化的键值对日志进行解析以及字段提取、路径转换、过滤等操作,去除无用字段的同时根据用户需求存储到指定路径,并将整个处理流程配置化
  \item 将处理后的日志分类存入HBase,并提供sdk和http两种查询方式,供用户使用,并提供配置化的存储查询方式
  \item 根据业务需求将日志根据id进行join操作,合并关联数据,实时的将数据发送给下游使用方,用户可根据提供的查询方式实时或离线获取所需日志信息
  \item 支持数据分流操作,针对同一种日志不同需求方可以获取不同的字段及存储路径,支持以配置的方式将数据实时写入kafka中供用户使用
  \item 提供id映射服务,实时的将用户手机号与系统版本、用户id等字段关联起来提供相关id的相互查询服务
\end{itemize}
\textbf{个人成果:}
\begin{itemize}
  \item 日志处理峰值可达27W+/s,各类数据存储量达50T,数据处理延迟保证在毫秒级以内
  \item 支撑多个团队所需日志数据,包括实时查询进行字段监控和离线查询进行问题定位
  \item ID映射服务被多个团队稳定使用,日均查询量在20W以上
  \item 以平台的形式提供相关日志查询及日志配置变更
\end{itemize}

\rule{\textwidth}{0.1mm}
\datedsubsection{\textbf{基于Spark技术的LTE网络信令数据实时分析系统研究}}{2014.09 -- 2016.06}
\role{\textbf{开发环境:}}{Eclipse + Ubuntu + MogoDB + Spark + Hadoop}
\role{\textbf{责任描述:}}{架构设计、平台实现、数据存储}
\begin{onehalfspacing}
\textbf{项目简介:}
\begin{itemize}
  \item 通过对LTE网络及协议的深入研究,提取信令数据特征
  \item 根据LTE信令数据海量和实时性特点,设计实现满足可扩展性以及实时性要求的Spark平台,实现数据的实时高效存储分析
  \item 利用LTE信令数据和Spark系统实现基于基站的用户实时定位,从而实现基于位置信息的网络优化
\end{itemize}
\textbf{项目成果:}
\begin{itemize}
  \item 针对基站定位系统已经基本实现上述功能,信令数据存入kafka消息队列由Spark Streaming消费,调用相关算法分别对每个用户的信息进行处理,将定位结果实时反馈,并存入HBase当中供后续查询处理
\end{itemize}
\end{onehalfspacing}

\section{\faInfo\ 其他}
% increase linespacing [parsep=0.5ex]
\begin{itemize}[parsep=0.5ex]
  \item 技术博客: http://www.cnblogs.com/gaopeng527
  \item 语\hspace{2em}言: 英语 - CET4
  \item 通讯地址:北京市昌平区龙跃苑4区40号楼6单元102
\end{itemize}

\section{\faTags\ 自我评价}
% increase linespacing [parsep=0.5ex]
\begin{itemize}[parsep=0.5ex]
  \item 工作认真负责,性格开朗,乐观向上
  \item 热爱软件事业,关注行业动态,对于IT领域的软件开发和工作有着浓厚的兴趣
  \item 吃苦耐劳、敢于面对和克服困难,能承受较大的工作压力
  \item 有良好的团队协调能力和环境适应能力,有责任感与使命感
\end{itemize}

%% Reference
%\newpage
%\bibliographystyle{IEEETran}
%\bibliography{mycite}
\end{document}
